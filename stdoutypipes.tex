\section{Uso de STDOUT y PIPES}
	\subsection{STDOUT}
		\noindent El operador \verb@a1 > a2@ redirecciona la salida estandar del comando a1 y la guarda en a2.
			\subsubsection[]{}
			\noindent El comando:\\
			 \verb@# ls -R /etc@ \\
			lista en forma recursiva el contenido del directorio \verb@/etc@.\\
			Con el caracter \verb@>@ redirecciona la salida de este comando a \verb@ /home/diego/tp/config@.\\
			El comando usado entonces es:\\
			 \verb@# ls -R /etc/ > /home/diego/tp/config@

			\subsubsection[]{}
			\noindent Con el comando:\\
			 \verb@# wc /home/diego/tp/config@\\
			la salida fue:\\
			\verb@682 610 7882 /home/diego/tp/config@\\
			El significado de esta salida es que el archivo tiene 682 lineas, 610 palabras y 7882 caracteres.

		  \subsubsection[]{}
		  \noindent Se ejecut'o el comando:\\
		    \verb@# sort /etc/passwd >> /home/diego/tp/config@\\
		  donde \verb@>>@  concatena detr'as de \verb@/home/diego/tp/config@ la salida (stdout) del comando sort.

		  \subsubsection[]{}
			\noindent Al ejecutar\\
			  \verb@# wc /home/diego/tp/config@ la salida fue:\\
			  \verb@705 640 8823 /home/diego/tp/config@\\
			El significado de esta salida es que el archivo tiene 705 lineas, 640 palabras y 8823 caracteres.

		\subsection{PIPES}
		\noindent El operador \verb@a1 | a2@ toma el STDOUT del comando a1 y lo envia al STDIN del comando a2.
		Ejecutando el siguiente comando:\\
		\verb@# ls -l /usr/bin/a* | grep apt | wc@\\
		se obtienen la cantidad de caracteres, palabras y lineas que vienen de filtrar el listado de los binarios con las lineas que contengan las palabra \verb@sync@.  La salida obtenida es:	\verb@12 96 853@
