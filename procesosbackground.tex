\section{Ejecuci'on de procesos en Background}
\subsection{Enunciado}
Cree el archivo /home/<usuario>/tp/loop.c. Comp'ilelo con \verb@gcc@. El programa compilado debe llamarse \verb@loop@. 
\begin{verbatim}
  #include <stdio.h>
  #define IDGRUPO 10;
  int main() {
    int i,c;
    while(1) {
      c = 48 + i;
      printf("%d", c);
      i++;
      i = i % IDGRUPO;
    }
    return 0;
  }
\end{verbatim}
\begin{enumerate}
\item Correrlo en foreground. ?`Qu'e suscede? Mate el proceso con \verb@Ctrl-c@
\item Ahora ejec'utelo en backgroun \verb|/usr/src/loop > /dev/null &|. Mate el proceso con el comando \verb@kill@
\end{enumerate}
\subsection{Desarrollo}
Usando el editor \verb@vi@ se cre'o el programa \verb@loop.c@ que se encuentra en la carpeta \verb@/home/grupo6/loop.c@.
Se compil'o el programa asign'andole a la macro \verb@IDGRUPO@ el valor 6. El comando ejecutado fue: \verb@$gcc loop.c -o loop@\\
\begin{asparaenum}
\item Al ejecutar el programa \verb@loop@ en foreground se pierde el control de esta terminal, ya que \verb@loop@ es un ciclo infinito. Cada vez que se ejecuta un comando en foreground, la consola queda esperando la finalizaci'on del mismo, siendo imposible realizar otra operaci'on en la misma consola.\\
Para recuperar el control de la terminal utilizamos la combinaci'on \verb@Ctrl-c@\\

\item Al ejecutar el programa con la sentencia \verb@$ ./loop > /dev/null &@ la salida standard del mismo es redireccionada a \verb@dev/null@. En definitiva se descarta la salida standard.\\
Con el \verb@&@ estamos indicando que el proceso debe ejecutarse en background. Al ejecutarse en background la consola queda libre para ejecutar cualquier comando a diferencia de la ejecuci'on en foreground, o sea, no espera a que el comando termine su ejecuci'on.\\
La ejecuci'on del programa con esta sentencia nos arroja como resultado inmediato el PID del proceso.\\
Utilizando el comando \verb@kill@ m'as el PID antes obtenido podemos matar el proceso.
\end{asparaenum}
