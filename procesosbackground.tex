\section{Ejecuci'on de procesos en Background}
\subsection{Enunciado}
\noindent Crear el siguiente programa
	\begin{verbatim}
		#include <stdio.h>
		int main()
		{
			int i,c;
			while(1)
			{
				c = 48 + i;
				printf("%d", c);
				i++;
				i = i % idgrupo;
			}
		}
	\end{verbatim}
Compilarlo. El programa compilado debe llamarse loop, Indicando a la macro idgrupo el valor
de su grupo.
\begin{itemize}
\item[a)] Correrlo en foreground. Qu'e sucede? Mate el proceso con el comando kill
\item[b)] Ahora ejec'utelo en background\\
            \verb|/usr/src/loop > /dev/null &|\\
            Que se muestra en la pantalla ?
\end{itemize}
Qu'e sucede si presiona la tecla F1? Qu'e significan esos datos ?\\
Qu'e sucede si presiona la tecla F2? Qu'e significan esos datos ?


\subsection{Desarrollo}
	Usando el editor \verb@vi@ se cre'o el programa \verb@loop.c@ que se encuentra en la carpeta \verb@/usr/pepeto/06@\\
	Se compil'o el programa asign'andole a la macro \verb@idgrupo@ el valor 6. El comando ejecutado fue:\\
	\verb@# cc loop.c -Didgrupo=6 -o loop@
	\begin{itemize}
	\item[a)] Al ejecutar el programa \verb@loop@ en foreground se pierde el control de esta terminal, ya que \verb@loop@ es un ciclo infinito. Cada vez que se ejecuta un comando en foreground, la consola queda esperando la finalizaci'on del mismo, siendo imposible realizar otra operaci'on en la misma consola.\\
		Para recuperar el control de la terminal, nos pasamos a otra terminal, ingresamos como usuario root al sistema y corremos el comando:\\
		 \verb@# ps -ax | grep loop@\\
		 el cual nos muestra la informaci'on relacionada con el proceso \verb@loop@. Anotamos su PID (n'umero indentificador de proceso) y ejecutamos\\
		 \verb@# kill 61@\\
		 donde 61 es el PID de \verb@loop@.\\
		Volvemos a la consola 1 observando que el proceso a finalizado mostrando la leyenda \verb@Terminated@.

	\item[b)] Al ejecutar el programa con la sentencia\\
		 \verb@# ./loop > /dev/null &@\\
		 la salida standard del mismo es redireccionada a \verb@dev/null@. En definitiva se descarta la salida standard.\\
		 Con el \verb@&@ estamos indicando que el proceso debe ejecutarse en background. Al ejecutarse en background la consola queda libre para ejecutar cualquier comando a diferencia de la ejecuci'on en foreground, o sea, no espera a que el comando termine su ejecuci'on.\\
		 La ejecuci'on del programa con esta sentencia nos arroja como resultado inmediato el PID del proceso.\\

		\textbf{Qu'e sucede si presiona la tecla F1?}\\
		Al presionar la tecla F1 se obtiene la siguiente tabla de procesos activos. 
		\begin{center}
			\includegraphics[width=1.00\textwidth]{images/f1.png}
		\end{center}

		Donde:

		\begin{itemize}
			\item[pid] Es el identificador de proceso.
			\item[sp] Es el puntero al stack del proceso.
			\item[flag] Es el registro flag del procesador.
			\item[user] Es el tiempo (de usuario) que lleva el proceso corriendo en el sistema.
			\item[text] Es el puntero al principio del segmento de c'odigo del proceso.
			\item[data] Es el puntero al principio del segmento de datos del proceso. Este valor coincide con el anterior pues en Minix los procesos comparten un bloque de memoria que se asigna y libera como una unidad.
			\item[Size] Es el tama~no de pila asignado al proceso.
			\item[command] Muestra el nombre del comando con el cual se invoc'o al proceso.
		\end{itemize}
		\textbf{Que sucede si presiona la tecla F2?}\\
		Al presionar la tecla F2 se obtiene el mapa de memoria de los procesos. 
		\begin{center}
			\includegraphics[width=1.00\textwidth]{images/f2.png}
		\end{center}
		Donde:

		\begin{itemize}
			\item[PROC] Es un n'umero asignado al proceso.
			\item[NAME] Es el nombre que poseen en el BCP.
			\item[TEXT, DATA, STACK] Muestran las entradas del arreglo \verb@mp_seg@ definido en \verb@/usr/src/mm/mproc.h@, la tabla de procesos del administrador de memoria. Cada entrada es una estructura que contiene la direcci'on virtual, la direcci'on f'isica y la longitud del segmento, todas medidas en clics (t'ipicamente 256 bytes).
			\item[SIZE] Es el tama~no que ocupa el proceso en memoria.
		\end{itemize}
	\end{itemize}
