\section{Temas del sistema operativo}
\subsection{File system}
\subsection{Prioridades}
\subsection{Par'ametros del Kernel}
La cantidad de memoria ram en el sistema puede ser consultada a traves el archivo \texttt{/proc/meminfo} a trav'es del comando \texttt{cat}. 'Este contiene informaci'on de la memoria, tanto f'isica como de intercambio, disponible en el sistema as'i como informaci'on de su uso. El primer campo que presenta en pantalla es \emph{MemTotal} que expresa la cantidad de memoria RAM del sistema expresada en kB. En el caso de nuestro sistema virtualizado estamos hablando de 250636 kB.

El kernel linux nos permite cambiar par'ametros en el momento de carga de la imagen a trav'es del bootloader. La lista de par'ametros disponibles se distribuye junto con las fuentes de linux y se encuentra detallada en \texttt{Documentation/kernel-parameters.txt}.

Para limitar la cantidad de memoria RAM visible al sistema se asigna el par'ametro \emph{mem} en el bootloader siguiendo la siguiente sintaxis:

\texttt{mem=nn[KMG]}

La misma documentaci'on advierte que al utilizarse en arquitecturas X86\_32 debe tambi'en definirse \emph{memmap} para evitar superposici'on de direcciones entre dispositivos PCI y memoria sin usar.
