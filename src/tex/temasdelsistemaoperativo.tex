\section{Temas del sistema operativo}
\subsection{File system}
Los \emph{hard links} son enlaces entre nombres simb'olicos e \textbf{inodos} del sistema del archivos. Linux, implementando el estandar POSIX, define en cada inodo la siguiente informaci'on:
\begin{itemize}
\item longitud (en bytes)
\item ID de usuario owner
\item ID de grupo al que pertenece el archivo
\item permisos de acceso al inodo
\item \emph{timestamps} referentes al acceso, creaci'on y modificaci'on del inodo
\item cantidad de enlaces a este inodo
\end{itemize}
Es interesante destacar que el espacio apuntado por un inodo es considerado como ocupado cuando tiene m'as de un enlace asociado a 'el. As'i uno puede crear un archivo de nombre foo, linkear un archivo bar a foo, luego eliminar foo y se puede seguir accediendo a la informaci'on original de foo a trav'es de bar; ese espacio queda asignado como archivo accesible hasta que desaparezca bar.
\subsection{Prioridades}
\subsection{Par'ametros del Kernel}
La cantidad de memoria ram en el sistema puede ser consultada a traves el archivo \texttt{/proc/meminfo} a trav'es del comando \texttt{cat}. 'Este contiene informaci'on de la memoria, tanto f'isica como de intercambio, disponible en el sistema as'i como informaci'on de su uso. El primer campo que presenta en pantalla es \emph{MemTotal} que expresa la cantidad de memoria RAM del sistema expresada en kB. En el caso de nuestro sistema virtualizado estamos hablando de 250636 kB.

El kernel linux nos permite cambiar par'ametros en el momento de carga de la imagen a trav'es del bootloader. La lista de par'ametros disponibles se distribuye junto con las fuentes de linux y se encuentra detallada en \texttt{Documentation/kernel-parameters.txt}.

Para limitar la cantidad de memoria RAM visible al sistema se asigna el par'ametro \emph{mem} en el bootloader siguiendo la siguiente sintaxis:

\texttt{mem=nn[KMG]}

La misma documentaci'on advierte que al utilizarse en arquitecturas X86\_32 debe tambi'en definirse \emph{memmap} para evitar superposici'on de direcciones entre dispositivos PCI y memoria sin usar.

\subsection{Administraci'on de memoria}
El tama~no de la partici'on \emph{swap} utilizada puede verse tambi'en como salida de \texttt{cat /proc/meminfo} en el par'ametro \textbf{SwapTotal:}. A su vez el detalle de los \emph{swaps} activos puede ser consultado a trav'es de \texttt{cat /proc/swaps}.

El tama~no de este archivo de intercambio puede ser incrementado utilizando el comando \texttt{swapon} sobre una partici'on o sobre un archivo fijo dentro de nuestro sistema de archivos. Para este 'ultimo, basta con crear un archivo del tama~no que nos interese incrementar la \emph{swap} mediante \texttt{dd if=/dev/zero of=[archivo-destino] bs=1024 count=[tama~no-deseado]}. Luego se le da formato de archivo de intercambio mediante \texttt{mkswap [archivo-destino]} y finalmente se informa al sistema operativo la existencia de este archivo para que lo utilice mediante \texttt{swapon [archivo-destino]}. Para que estos cambios queden persistentes en el entorno, se agrega en el archivo \texttt{/etc/fstab} la l'inea indicando el archivo, por ejemplo \texttt{[archivo-destino] none swap sw 0 0}. Este archivo se lee en el momento de booteo del sistema por el kernel y lo interpreta l'inea a l'inea para montar sectores de intercambio.
