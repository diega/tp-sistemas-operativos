\newenvironment{mylisting}
{\begin{list}{}{\setlength{\leftmargin}{1em}}\item\scriptsize\bfseries}
{\end{list}}

\newenvironment{mytinylisting}
{\begin{list}{}{\setlength{\leftmargin}{1em}}\item\tiny\bfseries}
{\end{list}}

\section{Comandos B'asicos de UNIX}
\subsection{\ttfamily{pwd}}
Muestra el nombre del directorio actual. Ejemplo:\\
      \begin{mylisting}
      \begin{verbatim}        
        $ cd /usr/bin/
        $ pwd
        /usr/bin
        $ cd
        $ pwd
        /home/grupoXX
      \end{verbatim}
      \end{mylisting}
Esto ocurre porque el comando \texttt{cd} invocado sin par'ametros establece como directorio actual el indicado por la variable de entorno \texttt{\$HOME}.

\subsection{\ttfamily{cat}}
Concatena archivos y escribe el resultado en la salida estandar \mbox{(stdout)}. La salida obtenida es:\\
      \begin{mylisting}
      \begin{verbatim}
        $ cat /home/grupoXX/.profile
        # ~/.profile: executed by the command interpreter for login shells.
        # This file is not read by bash(1), if ~/.bash_profile or 
        # ~/.bash_login exists.
        # see /usr/share/doc/bash/examples/startup-files for examples.
        # the files are located in the bash-doc package.

        # the default umask is set in /etc/profile
        #umask 022

        # if running bash
        if [ -n "$BASH_VERSION" ]; then
            # include .bashrc if it exists
            if [ -f "$HOME/.bashrc" ]; then
          . "$HOME/.bashrc"
            fi
        fi

        # set PATH so it includes user's private bin if it exists
        if [ -d "$HOME/bin" ] ; then
            PATH="$HOME/bin:$PATH"
        fi
      \end{verbatim}
      \end{mylisting}
El archivo \texttt{/home/grupoXX/.profile} se utiliza para personalizar el entorno de ejecuci'on. Usualmente aqu'i se define la variable de entorno \texttt{PATH}, una variable cuyo valor indica los directorios d'onde Linux buscar'a cuando se ejecute un comando sin explicitar la ruta de acceso a 'el.

Tambi'en ejecuta el archivo \texttt{.bashrc} para cargar las configuraciones propias del int'erprete de comandos \texttt{.bash}.

\subsection{\ttfamily{find}}
Busca archivos en la estructura de directorios.

Para obtener el path de los archivos que comienzan con \texttt{vmlinuz}, podemos invocar el comando \texttt{find} de la siguiente manera:
      \begin{mylisting}
      \begin{verbatim}
	sudo find / -name 'vmlinuz*' -print
      \end{verbatim}
      \end{mylisting}
De esta forma, se indica que busque los archivos recursivamente desde el directorio \texttt{/}, que adem'as se desea buscar los archivos cuyo nombre comiencen \texttt{vmlinuz}, y que los resultados se deben mostrar por la salida est'andar (si no se escribe \texttt{-print}, igualmente se muestra el resultado por salida est'andar).
La salida obtenida es:
      \begin{mylisting}
      \begin{verbatim}
	/vmlinuz
	/boot/vmlinuz-2.6.28-11-virtual
      \end{verbatim}
      \end{mylisting}

\subsection{\ttfamily{mkdir}}
Para crear un directorio en el directorio \texttt{/home/grupoXX} el comando se debe invocar de la siguiente manera:

\texttt{mkdir /home/grupoXX/tp}

\subsection{\ttfamily{cp}}
Para copiar el archivo \texttt{/etc/passwd} al directorio \texttt{/home/grupoXX/tp}, se debe escribir la siguiente instrucci'on:

\texttt{cp /etc/passwd /home/grupoXX/tp}

\subsection{\ttfamily{chgrp}}
Para cambiar el grupo al que pertenece el archivo \texttt{/home/grupoXX/tp/passwd} y establecerlo en \emph{grupoXX} se debe ejecutar:

\texttt{chgrp grupoXX /home/grupoXX/tp/passwd}

\subsection{\ttfamily{chown}}
Para cambiar el propietario al que pertenece el archivo \texttt{/home/grupoXX/tp/passwd} y establecerlo en \emph{grupoXX} se debe ejecutar:

\texttt{chown grupoXX /home/grupoXX/tp/passwd}

\subsection{\ttfamily{chmod}}
Para que el propietario tenga permisos de lectura, escritura y ejecuci'on sobre el archivo \texttt{/home/grupoXX/tp/passwd} se debe invocar:

\texttt{chmod u+rwx /home/grupoXX/tp/passwd}

Para que el grupo tenga solo permisos de lectura y ejecuci'on sobre el archivo \texttt{/home/grupoXX/tp/passwd}:

\texttt{chmod g+rx-w /home/grupoXX/tp/passwd}

Es decir, se debe sacar el permiso de escritura.
Para que el resto (los que no son ni el propietario ni miembros del grupo) tengan solo permisos de ejecuci'on sobre el archivo \texttt{/home/grupoXX/tp/passwd} se debe invocar:

\texttt{chmod o-r-w+x /home/grupoXX/tp/passwd}

\subsection{\ttfamily{grep}}
Para visualizar las lineas que tienen el texto localhost en el archivo /etc/hosts se ejecuta el comando

\texttt{grep localhost /etc/hosts}

La salida obtenida es:
      \begin{mylisting}
      \begin{verbatim}
	127.0.0.1     localhost
	::1           ip6-localhost ip6-loopback
      \end{verbatim}
      \end{mylisting}
Para mostrar las lineas que tiene el texto POSIX que se encuentren en todos los archivos dentro de \texttt{/etc/}, excluyendo los binarios y los que no pertenecen al usuario, se debe ejecutar el comando \texttt{grep -r -I -s POSIX /etc/*}.
Para que que s'olo muestre el nombre y la ruta de los archivos que contienen el texto POSIX se agrega el modificador -l:

\texttt{grep -r -I -s -l POSIX /etc/*}

La salida obtenida es:
      \begin{mylisting}
      \begin{verbatim}
	/etc/init.d/glibc.sh
	/etc/security/limits.conf
      \end{verbatim}
      \end{mylisting}

\subsection{\ttfamily{passwd}}
\subsubsection[]{Cambie su password}
Para cambiar el password del usuario logueado se debe ejecutar el comando \texttt{passwd} sin par'ametros, el cual pide que ingrese el password actual (por cuestiones de seguridad), el nuevo seguido de su confirmaci'on. 

\subsection{\ttfamily{rm}}
Para borrar el archivo \texttt{/home/grupoXX/tp/passwd} se debe invocar \texttt{rm} de la siguiente manera:

\texttt{rm /home/grupoXX/tp/passwd}

\subsection{\ttfamily{ln}}
Para crear dos enlaces \emph{hard} al archivo \texttt{/etc/passwd} llamados \texttt{/tmp/contra1} y \texttt{/tmp/contra2}, se debe ejecutar dos veces el comando \texttt{ln} de la siguiente manera:

\texttt{ln /etc/passwd /tmp/contra1}

\texttt{ln /etc/passwd /tmp/contra2}

Para poder chequear cuantos enlaces tiene el archivo \texttt{/etc/passwd}:

\texttt{ls -l /etc/passwd}

La salida obtenida es:

\texttt{-rw-r--r--  3 root     root      941 2008-08-30 12:18 /etc/passwd}

El n'umero 3 indica que hay 3 nombres que referencian a un mismo archivo en memoria. Si se borrase el archivo \texttt{/etc/passwd}, lo que se estar'ia borrando ser'ia una referencia al archivo y no el archivo en s'i. La manera de borrar el archivo f'isicamente es eliminando todas las referencias (o \emph{hardlinks}) que posea.

Para generar un \emph{soft link} corremos \texttt{ln -s /etc/passwd /tmp/contra3} y se chequea que efectivamente el n'umero 3 antes obtenido sigue igual.

\subsection{\ttfamily{mount}}
Para montar el CD-ROM en un directorio (por ejemplo \texttt{/media/cdrom} ejecutamos:

\texttt{mount /dev/scd0 /media/cdrom}

La salida es:

\texttt{mount: block device /dev/scd0 is write-protected, mounting read-only}
		
Para ver los filesystems montados en el sistema, se corre el comando \texttt{mount} sin par'ametros.
      \begin{mylisting}
      \begin{verbatim}
	/dev/sda1 on / type ext3 (rw,relatime,errors=remount-ro)
	proc on /proc type proc (rw,noexec,nosuid,nodev)
	/sys on /sys type sysfs (rw,noexec,nosuid,nodev)
	varrun on /var/run type tmpfs (rw,noexec,nosuid,nodev,mode=0755)
	varlock on /var/lock type tmpfs (rw,noexec,nosuid,nodev,mode=1777)
	udev on /dev type tmpfs (rw,mode=0755)
	devshm on /dev/shm type tmpfs (rw)
	devpts on /dev/pts type devpts (rw,gid=5,mode=620)
	/dev/scd0 on /media/cdrom0 type iso9660 (ro)
      \end{verbatim}
      \end{mylisting}

\subsection{\ttfamily{df}}
Para ver informaci'on sobre los filesystems montados en el sistema se invoca el comando \texttt{df} sin par'ametros.

La salida del comando es:
       \begin{mylisting}
       \begin{verbatim}
	Filesystem           1K-blocks      Used Available Use% Mounted on
	/dev/sda1               505604    377943    101557  79% /
	varrun                  127884        28    127856   1% /var/run
	varlock                 127884         0    127884   0% /var/lock
	udev                    127884        40    127844   1% /dev
	devshm                  127884         0    127884   0% /dev/shm
	/dev/scd0               101916    101916         0 100% /media/cdrom0
       \end{verbatim}
       \end{mylisting}

\subsection{\ttfamily{ps}}
\subsubsection[]{Cu'antos procesos de usuario tiene ejecutando?}
Para obtener una lista de los procesos del usuario se invoca el comando \texttt{ps}:

La salida obtenida es:
       \begin{mylisting}
       \begin{verbatim}
	PID TTY       TIME CMD
	3694 tty1  00:00:00 bash
	12613 tty1  00:00:00 ps
      \end{verbatim}
      \end{mylisting}
Por lo tanto, hay 2 procesos de usuario corriendo actualmente.

\subsubsection[]{Indique cuantos son del sistema.}
Para obtener una lista de todos los procesos (usuario y sistema) se invoca el comando \texttt{ps} indicando como argumento \texttt{-A}:

La salida obtenida es:
      \begin{mylisting}
      \begin{verbatim}
	  PID TTY          TIME CMD
	    1 ?        00:00:01 init
	    2 ?        00:00:00 kthreadd
	    3 ?        00:00:00 migration/0
	    4 ?        00:00:00 ksoftirqd/0
	    5 ?        00:00:00 watchdog/0
	    6 ?        00:00:00 events/0
	    7 ?        00:00:00 khelper
	   36 ?        00:00:00 kblockd/0
	   39 ?        00:00:00 kacpid
	   40 ?        00:00:00 kacpi_notify
	   79 ?        00:00:00 kseriod
	  111 ?        00:00:00 pdflush
	  112 ?        00:00:00 kswapd0
	  153 ?        00:00:00 aio/0
	 1208 ?        00:00:00 ata/0
	 1211 ?        00:00:00 ata_aux
	 1228 ?        00:00:00 scsi_eh_0
	 1232 ?        00:00:00 scsi_eh_1
	 2089 ?        00:00:01 kjournald
	 2230 ?        00:00:01 udevd
	 2425 ?        00:00:00 kpsmoused
	 3295 ?        00:00:00 dhclient3
	 3604 tty4     00:00:00 getty
	 3607 tty5     00:00:00 getty
	 3612 tty2     00:00:00 getty
	 3617 tty3     00:00:00 getty
	 3620 tty6     00:00:00 getty
	 3653 ?        00:00:00 syslogd
	 3672 ?        00:00:00 dd
	 3674 ?        00:00:00 klogd
	 3693 tty1     00:00:00 login
	 3694 tty1     00:00:00 bash
	12202 ?        00:00:00 pdflush
	12648 tty1     00:00:00 ps
      \end{verbatim}
      \end{mylisting}
Por lo tanto, la cantidad de procesos del sistema es:

Cantidad total de procesos = Cantidad total de procesos totales - Cant de procesos del usuario =  35 - 2 = 33

\subsection{\ttfamily{umount}}
Desmonte el CD-ROM del directorio \texttt{umount /media/cdrom}

Se ejecuta el comando \texttt{umount /dev/scd0}

\subsection{\ttfamily{uptime}}
\texttt{uptime} indica cu'anto tiempo hace que est'a el sistema corriendo:

\texttt{11:11:46 up  2:43,   1 user,   load average: 0.00, 0.00, 0.00}

\subsection{\ttfamily{uname}}
\texttt{uname} imprime informaci'on sobre el kernel que estamos corriendo. Se lo corre con la opci'on \texttt{-r} para que imprima sola la versi'on de, en este caso, linux.

\texttt{2.6.24-19-virtual}
