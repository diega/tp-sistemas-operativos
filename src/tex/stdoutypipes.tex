\section{Uso de STDOUT y PIPES}
\subsection{STDOUT}
El operador \texttt{a1 > a2} redirecciona la salida estandar del comando a1 y la guarda en el archivo a2.

\begin{enumerate}[a)]

\item El comando \texttt{ls -R /etc} lista en forma recursiva el contenido del directorio \texttt{/etc}.
Con el operador \texttt{>} se redirecciona la salida de este comando a \texttt{/home/diego/tp/config}.

El comando usado entonces es: \texttt{ls -R /etc/ > /home/diego/tp/config}

\item Con el comando: \texttt{wc /home/diego/tp/config} la salida fue:

\texttt{682 610 7882 /home/diego/tp/config}

El significado de esta salida es que el archivo tiene 682 lineas, 610 palabras y 7882 caracteres.

\item Se ejecut'o el comando \texttt{sort /etc/passwd $>>$ /home/diego/tp/config} donde \texttt{$>>$} concatena detr'as de \texttt{/home/diego/tp/config} la salida (stdout) del comando sort.

\item Al ejecutar \texttt{wc /home/diego/tp/config} la salida fue:

\texttt{705 640 8823 /home/diego/tp/config}

El significado de esta salida es que el archivo tiene 705 lineas, 640 palabras y 8823 caracteres.
\end{enumerate}

\subsection{PIPES}
\begin{enumerate}[a)]
\item El operador \texttt{a1 | a2} toma el STDOUT del comando a1 y lo envia al STDIN del comando a2.

Ejecutando el comando \texttt{ls -l /usr/bin/a* | grep apt | wc} se obtienen la cantidad de caracteres, palabras y lineas que vienen de filtrar el listado de los binarios con las lineas que contengan la letra \texttt{a}.

La salida obtenida es: \texttt{12 96 853}
\end{enumerate}
