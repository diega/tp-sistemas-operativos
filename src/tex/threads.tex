\section{Ejecuci'on de procesos en Background}
\subsection{Enunciado}
\subsubsection{Threads}

Para este ejercicio realizamos 4 m'etodos. El productor (produce), el que llena el buffer con lo producido (producer), un consumidor (consume) y el que consume los datos del buffer (consumer).
\\
\\ Pasemos a ver cada uno:
\\
\\ \textbf{producer:}
\\
\\Producer recibe el buffer donde pondr'a los datos producidos por el productor (produce).
\\ Para esto usa un ciclo en el cual va leyendo los datos del archivo del cual lee y luego llama a produce para que los valla poniendo en el buffer. Cuando llega al final del archivo termina.
\\
\\ \textbf{produce:}
\\
\\Produce recibe el buffer y el caracter a escribir en el buffer. Con esos datos pasa a ver si puede escribir en el buffer haciendo un P(buffer.lock) (para esto usa el pthread$\_mutex\_lock$) y si no puede escribir se lockea.
\\ Si puede producir lo indica imprimiendo "Produce" y pasa a intentar llenarlo. Para eso entra a un ciclo en el que se va fijando si el buffer esta lleno (b.occupied == BSIZE), si lo est'a espera hasta ser despertado (para esto usa el $pthread\_cond\_wait$).
\\ Una vez ya puede escribir el caracter en el buffer lo hace. Para esto primero lo escribe y adelanta la posici'on del buffer a escribir (nextin) fij'andose que no sobrepase la longitud del mismo, y si lo hace, entonces le toma el m'odulo a la longitud (BSIZE). Y finalmente le suma uno m'as a la cantidad de posiciones ocupadas (occupied) para luego saber cuando esta totalmente lleno.
\\ Por 'ultimo hace un signal para despertar a los consumidores que estuvieran dormidos por no haber encontrado datos en el buffer (usa para esto $pthread\_cond\_signal$ sobre el more del buffer) y luego deslockea nuevamente el buffer con un V(buffer.lock) (usando $pthread\_mutex\_unlock$).
\\nextin va movi'endose circularmente sobre el buffer dado que la pol'itica de producci'on y consumi'on es FIFO. Y por eso es necesaria la variable occupied para saber si el buffer est'a lleno o no.
\\
\\ \textbf{consumer:}
\\
\\Consumer crea una variable para almacenar el caracter a leer del buffer, indica que va a consumir caracteres para el proceso del que obtiene el id (con getpid).
\\Y pasa a pedir, dentro de un ciclo, a consume que consuma del buffer que le pasa. Consume le va devolviendo caracter a caracter lo consumido y mientras este no sea el final sigue leyendo.
\\
\\\textbf{consume:}
\\
\\Consume recibe el buffer y devuelve el caracter que se consumi'o del buffer. Ya teniendo el buffer trata de cerrarlo haciendo un P() del sem'aforo del buffer (con $pthread\_mutex\_lock$) , si no estaba lockeado de antes, etonces podr'a consumir. Sino tendr'a que esperar.
\\El paso siguiente es ver si hay datos para consumir en el buffer para lo que usa un ciclo en que se fija si es el buffer esta vac'io (b.occupied == 0) y si lo est'a se quedara en espera de que alg'un productor lo llene. Esto lo hace usando el $pthread\_mutex\_lock$.
\\ Por el contrario, si puede consumir lo indica imprimiendo "Consume" y pasa a consumirlo. Para esto primero lo lee de la posici'on nextout, la cual luego la apunta a la siguiente posici'on a consumir.
\\A continuaci'on decrementa la cantidad de posiciones ocupadas del buffer (decrementando occupied) y se fija que no pase a valer menos de cero, lo cual lo comprueba haciendo el m'odulo de la misma con la longitud del mismo(BSIZE). Con esto 'ultimo sabr'a cuando el buffer est'e vac'io.
\\ Por 'ultimo hace un signal para despertar a los productores que estuvieran dormidos por no haber podido producir por haber estado lleno el buffer (usa para esto $pthread\_cond\_signal$ sobre el less del buffer) y luego deslockea nuevamente el buffer con un V(buffer.lock) (usando $pthread\_mutex\_unlock$).
\\nextou va movi'endose circularmente sobre el buffer dado que la pol'itica de producci'on y consumi'on es FIFO. Y por eso es necesaria la variable occupied para saber si el buffer est'a lvac'io o no.
\\
\\\textbf{Inicializaciones necesarias:}
\\
\\Para que funcione correctamente los algoritmos es necesario incluir los paquetes pthread.h y stdio.h. Definir a la longitud del buffer como variable global y crear la estructura del tipo $buffer\_t$ que vamos a usar como buffer.
\\
\\typedef struct $buffer\_t$ \{
\\\indent    char buf[BSIZE];
\\\indent    $pthread\_mutex\_t$ lock;
\\\indent    $pthread\_cond\_t$ less;
\\\indent    $pthread\_cond\_t$ more;
\\\indent    int nextin;
\\\indent	 int nextout;
\\\indent	 int occupied;
\\\} $buffer\_t$;
\\
\\\textbf{Y en el main inicializamos lo datos necesarios:}

\begin{verbatim}

	buffer_t buffer;
	pthread_mutexattr_t mutex_attr;
	pthread_condattr_t cond_attr;
	buffer.occupied = buffer.nextin = buffer.nextout = 0;
	
	pthread_mutexattr_init(&mutex_attr);
	pthread_mutexattr_setpshared(&mutex_attr, PTHREAD_PROCESS_SHARED);
	pthread_mutex_init(&buffer.lock, &mutex_attr);
	
	pthread_condattr_init(&cond_attr);
	pthread_condattr_setpshared(&cond_attr, PTHREAD_PROCESS_SHARED);
	pthread_cond_init(&buffer.less, &cond_attr);
	pthread_cond_init(&buffer.more, &cond_attr);
\end{verbatim}

Y llamamos a los procesos dividiendo ambas opciones con un fork.
\\
\begin{verbatim}
	if (fork() == 0) {
		consumer(buffer);
		return 0;
	}
	else {
		producer(buffer);
		return 0;
	}
\end{verbatim}

Para compilar, guardamos el c�digo como threads.c y lo compilamos por linea de comando con lo siguiente:
\\gcc -static threads.c -o threads -L -ISDL -lm -lpthread
\\

%\~n