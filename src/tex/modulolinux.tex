\section{M'odulos del kernel linux}
\subsection{Enunciado}
Escriba un m'odulo de kernel que permita controlar los LEDs del teclado sin necesidad de escribir un programa en lenguaje C. El m'odulo deber'a permitir prender o apagar cada LED usando simplemente comandos del shell.
\subsection{Desarrollo}
Al ser Linux un kernel monol'itico provee como mecanismo de extensi'on el desarrollo de m'odulos. 'Estos viven en el mismo espacio de memoria que el n'ucleo y se comportan exactamente igual que si hubieran sido generados mediante el mismo proceso de compilaci'on ocupando el mismo binario. Por este motivo tienen acceso total a las estructuras internas de manera directa y no mediante \emph{system calls}. Cabe destacar que NO tienen acceso a ninguna librer'ia externa, no se puede usar nada que venga del espacio del usario, como por ejemplo glibc.

Para el fin de el m'odulo pedido cre'e un archivo dentro de proc para ser escrito desde el espacio de usuario, el cual disparar'a una \emph{system call} atrapada por el kernel y delegada a la implementaci'on nuestra, que es quien se encarga de llamar a la funci'on para prender cada LED seg'un lo que se haya volcado al archivo. Se utiliza un protocolo muy b'asico para la interacci'on con el m'odulo. El archivo creado se llama \textbf{keyboard\_leds}, en 'el se podr'a escribir el caracter 1 para encender el Num Lock, 2 para encender Caps Lock y 3 para Scroll Lock cualquier otro caracter ingresado, as'i como cualquier cadena que m'as all'a del primer caracter, ser'a ignorado.

Respecto de la implementaci'on del mismo, se hizo uso de la funci'on \texttt{create\_proc\_entry} la que recibiendo como par'ametros el nombre del archivo y los permisos y nos devuelve una estructura con informaci'on para su manipulaci'on, es decir las acciones que ser'an disparadas frente a distintos \emph{est'imulos} al archivo. En particular, a la funci'on asignada para ser llamada frente a la escritura se le pasa como par'ametro la estructura \texttt{file} sobre la que se est'a ejecutando la acci'on, un buffer existente en el espacio de usuario que contiene lo que se ha escrito y la longitud de lo almacenado en el buffer. La memoria residente en el espacio de usuario no puede ser le'ida directamente sino que debe ser a trav'es del m'etodo \texttt{copy\_from\_user} que lo coloca disponible para su utilizaci'on dentro del m'odulo. Una vez obtenido el contenido del archivo utilizando la referencia a la terminal virtual correspondiente, se invoca la \emph{ioctl} para trabajar con los LEDs del teclado, y voil'a...
\subsection{C'odigo Fuente}
    \lstinputlisting[language=C,breaklines]{../module/keyboard_leds_module.c}
