\section{IPC y sincronizaci'on}
Ambos archivos fuente pueden encontrarse en la carpeta src/c y se compilan mediante la llamada a \texttt{make}.
\subsection{Pipes}
Para crear un archivo de pipe utilizamos la \textit{system call} \texttt{pipe()}. Esta toma un solo argumento que es un array de dos enteros que, luego de correrla y si no ocurre alg'un error, contendr'a dos \textit{file descriptors} para usar en el \textit{pipeline} (o tuber'ia). El primer entero en el array (elemento 0) est'a abierto para ser le'ido, mientras que el segundo entero (elemento 1) lo est'a para ser escrito; nos referiremos usualmente a estos archivos como fd0 y fd1. En una representaci'on m'as visual podr'iamos decir que la salida de fd1 se convierte en la entrada de fd0.

Si usamos pipes para comunicar procesos quien quiera leer del pipe debe cerrar fd1 y quien escriba debe hacer lo mismo con fd0. Es importante siempre cerrar el extremo del pipe que no estemos utilizando.

Luego de crear el pipe, mediante la instrucci'on \texttt{fork()}, creamos un nuevo thread sabiendo que estos heredan los file descritors abiertos de sus padres. Con esta facilidad la comunicaci'on es sencilla.

La operaci'on de lectura sobre el pipe quedar'a bloqueada hasta que alguien escriba en 'el. Este comportamiento es explotado por nosotros para desarrollar la mutua exclusi'on sobre alguna zona cr'itica.

Nuestra implementaci'on particular crea dos threads permitiendo que el primero ejecute la zona cr'itica, cuya finalizaci'on ser'a informada al hermano mediante la escritura del pipe. 
\lstinputlisting[language=C,breaklines]{../c/pipes.c}

\subsection{Threads}
Para este ejercicio realizamos 4 m'etodos. El productor (produce), el que llena el buffer con lo producido (producer), un consumidor (consume) y el que consume los datos del buffer (consumer).

\subsubsection{Descripci'on de los m'etodos para la resoluci'on}

\paragraph{producer}
Producer recibe el buffer donde pondr'a los datos producidos por el productor (produce). Para esto usa un ciclo en el cual va leyendo los datos del archivo del cual lee y luego llama a produce para que los valla poniendo en el buffer. Cuando llega al final del archivo termina.

\paragraph{produce}
Produce recibe el buffer y el caracter a escribir en el buffer. Con esos datos pasa a ver si puede escribir en el buffer haciendo un P(buffer.lock) (para esto usa el pthread\_mutex\_lock) y si no puede escribir se lockea.  Si puede producir lo indica imprimiendo "Produce" y pasa a intentar llenarlo. Para eso entra a un ciclo en el que se va fijando si el buffer esta lleno (b.occupied == BSIZE), si lo est'a espera hasta ser despertado (para esto usa el pthread\_cond\_wait). Una vez ya puede escribir el caracter en el buffer lo hace. Para esto primero lo escribe y adelanta la posici'on del buffer a escribir (nextin) fij'andose que no sobrepase la longitud del mismo, y si lo hace, entonces le toma el m'odulo a la longitud (BSIZE). Y finalmente le suma uno m'as a la cantidad de posiciones ocupadas (occupied) para luego saber cuando esta totalmente lleno. Por 'ultimo hace un signal para despertar a los consumidores que estuvieran dormidos por no haber encontrado datos en el buffer (usa para esto pthread\_cond\_signal sobre el more del buffer) y luego deslockea nuevamente el buffer con un V(buffer.lock) (usando pthread\_mutex\_unlock). nextin va movi'endose circularmente sobre el buffer dado que la pol'itica de producci'on y consumi'on es FIFO. Y por eso es necesaria la variable occupied para saber si el buffer est'a lleno o no.

\paragraph{consumer}
Consumer crea una variable para almacenar el caracter a leer del buffer, indica que va a consumir caracteres para el proceso del que obtiene el id (con getpid). Y pasa a pedir, dentro de un ciclo, a consume que consuma del buffer que le pasa. Consume le va devolviendo caracter a caracter lo consumido y mientras este no sea el final sigue leyendo.

\paragraph{consume}
Consume recibe el buffer y devuelve el caracter que se consumi'o del buffer. Ya teniendo el buffer trata de cerrarlo haciendo un P() del sem'aforo del buffer (con pthread\_mutex\_lock) , si no estaba lockeado de antes, etonces podr'a consumir. Sino tendr'a que esperar. El paso siguiente es ver si hay datos para consumir en el buffer para lo que usa un ciclo en que se fija si es el buffer esta vac'io (b.occupied == 0) y si lo est'a se quedara en espera de que alg'un productor lo llene. Esto lo hace usando el pthread\_mutex\_lock. Por el contrario, si puede consumir lo indica imprimiendo "Consume" y pasa a consumirlo. Para esto primero lo lee de la posici'on nextout, la cual luego la apunta a la siguiente posici'on a consumir. A continuaci'on decrementa la cantidad de posiciones ocupadas del buffer (decrementando occupied) y se fija que no pase a valer menos de cero, lo cual lo comprueba haciendo el m'odulo de la misma con la longitud del mismo(BSIZE). Con esto 'ultimo sabr'a cuando el buffer est'e vac'io. Por 'ultimo hace un signal para despertar a los productores que estuvieran dormidos por no haber podido producir por haber estado lleno el buffer (usa para esto pthread\_cond\_signal sobre el less del buffer) y luego deslockea nuevamente el buffer con un V(buffer.lock) (usando pthread\_mutex\_unlock). nextout va movi'endose circularmente sobre el buffer dado que la pol'itica de producci'on y consumi'on es FIFO. Y por eso es necesaria la variable occupied para saber si el buffer est'a vac'io o no.

\subsubsection{Otras consideraciones de la resoluci'on}
Para que funcione correctamente es necesario incluir los paquetes pthread.h y stdio.h, definir la longitud del buffer como variable global y crear la estructura del tipo buffer\_t que vamos a usar como buffer.
\begin{mylisting}
\begin{verbatim}
typedef struct buffer_t {
	char buf[BSIZE];
	pthread_mutex_t lock;
	pthread_cond_t less;
	pthread_cond_t more;
	int nextin;
	int nextout;
	int occupied;
} buffer_t;
\end{verbatim}
\end{mylisting}
Y en el main inicializamos lo datos necesarios:
\begin{mylisting}
\begin{verbatim}
        buffer_t buffer;
        pthread_mutexattr_t mutex_attr;
        pthread_condattr_t cond_attr;
        buffer.occupied = buffer.nextin = buffer.nextout = 0;
        
        pthread_mutexattr_init(&mutex_attr);
        pthread_mutexattr_setpshared(&mutex_attr, PTHREAD_PROCESS_SHARED);
        pthread_mutex_init(&buffer.lock, &mutex_attr);
        
        pthread_condattr_init(&cond_attr);
        pthread_condattr_setpshared(&cond_attr, PTHREAD_PROCESS_SHARED);
        pthread_cond_init(&buffer.less, &cond_attr);
        pthread_cond_init(&buffer.more, &cond_attr);
\end{verbatim}
\end{mylisting}
Y llamamos a los procesos dividiendo ambas opciones con un fork.
\begin{mylisting}
\begin{verbatim}
        if (fork() == 0) {
                consumer(buffer);
                return 0;
        }
        else {
                producer(buffer);
                return 0;
        }
\end{verbatim}
\end{mylisting}
La implementaci'on completa a continuaci'on
\lstinputlisting[language=C,breaklines]{../c/threads.c}
